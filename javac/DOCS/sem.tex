\NeedsTeXFormat{LaTeX2e}[1994/06/01]
\documentclass{article}[1994/05/24]

\title{sem: A Semantic Analyzer for Java}

\author{Michael Schaeffer}
\date{November 11, 1996}

\begin{document}


\maketitle

%%%%%%%%%%%%%%%%%%%%%%%%%%%%%%%%%%%%%%%%%%%%%%%%%%%%%%%%%%%%%%%%
%%%%%%%%%%%%%%%%%%%%%%%%%%%%%%%%%%%%%%%%%%%%%%%%%%%%%%%%%%%%%%%%
%%%%%%%%%%%%%%%%%%%%%%%%%%%%%%%%%%%%%%%%%%%%%%%%%%%%%%%%%%%%%%%%

\section{Introduction}

	The \verb|sem| phase of this compiler is the final step the
compiler takes towards extracting information from the source
file. Unlike the other phases, where the analysis was entirely
directed by the physical description of the language, in semantic
analysis, the actual meaning behind the structure is taken into 
consideration. As a result, the \verb|sem| phase has to not only be
able to access the entire data carried in the source file (through
the ast), but also be able to resolve any ambiguities contained
therein.

\section{Differences from the Java Specification}

	Unlike the lex and syn phases, the \verb|sem| phase makes
no attempt at covering the entire scope of the language.  The 
subset decisions made while developing this phase drastically 
alter the language. The good news is that the \verb|sem| phase
does fully support atomic types, expressions, function calls,
classes, arrays, and inheritance. It easily supports the
intersection between java and C, and goes beyond that to
support a simple form of OOP. \verb|sem|'s main shortcoming is that
it  doesn't support importing other package files.  As a result,
many of the basic classes needed to support more advanced
features like strings, threads, and exceptions are not available.
\verb|sem| also lacks support for type modifiers and initialized
variables.

\section{Theory of Operation}

	\verb|sem|'s modus operandi is the tree transformation. All of
the significant computations performed by \verb|sem| are done by
walking up and down the ast, destructively modfying the structure to
transform it to be more representitive of the program.  These tree
transformations are administered by a function \verb|traverse-ast|.
\verb|traverse-ast| takes an ast node and an operation code as input
paramaters.  It then looks up the appropriate operation to take for
the given node type and action code.  In this manner, \verb|traverse-ast| 
is both paramater and data directed.

	To fully process an ast, \verb|sem| runs three seperate
transformations: normalize, parent-scope, and type-check, in that
sequence.  The normalize traversal's responsibility is to massage the
ast from \verb|syn| into a more managable form.  It moves declarations
into thier own attribute, it simplifies the representation of types in 
the ast, and it allocates unique numbers for variable declarations.  The
parent-scope phase, executed next, is by far the simplest traversal.  It
traverses the tree, and sets up links from child nodes to parent nodes.
This will be used in the type-check phase to help resolve symbol names.

	The final, and most complicated traversal, is the type checker.
The type checker's main responsibility is to bind variable declarations
to variable instances, and then to propagate type data up the tree.
The type checker tries to resolve inconsistancies in the ast by
inserting type coercion nodes according to what is allowed by the
type system. Any type inconsistancies that cannot be resolved through
automatic coercion are reported to the user as an error and processing 
stops. The type checker also maintains relationships between classes
and base classes

\section{The AST, the Symbol Table, and the Type Table}

	\verb|sem| contains no explicit symbol table.  Instead, the
symbol table is interwoven with the ast, with symbol data stored
in the \verb|'SYMBOLS| attribute of each appropriate node.  For functions,
the symbol definition includes the function body.  In this manner, there
are symbol tables within symbol tables. To resolve a name, the
function \verb|retrieve-symbol| walks up the tree, following the
parent link, until it finds the first instantce of a symbol. This allows
each level in the parse tree to have its own scope level. Inheritance
is handled by linking the base class' symbol table onto the end of
the sub class.

\section{Interpreting Lisp dumps of the ast}

Like the output from \verb|syn|, each node in the ast has the following
form:

\begin{verbatim}
(<node-type> ((<attr-name> . <attr-value> ...) <node-child> ...))
\end{verbatim}

However, unlike the \verb|syn| ast, in \verb|sem|, much data is stored
in attributes.  Some common attrubute names are:

\begin{description}

\item[name] The name of the object being definied

\item[code] The code of the object being defined

\item[type] The resultant type of the ast node's operation

\item[usage] The usage of the node's result (l-value, r-value, or lr-value)

\item[parent] A pointer to the parent node

\item[symbols] The symbol table of a given ast node.

\item[unique] A pointer to the \verb|gensym| attached to a declaration

\end{description}

Since the ast is now a circular structure, the Lisp print routine
takes precautions to avoid being stuck in an infinite recursion. 
Anytime there is a pointer to something that has already been printed,
the printer prints out something of the form \verb|#n#|.  This means
to substitute the definition \verb|#n=<defn>| in its place.  This is
particularly evident when looking at the unique numbers for variable
instances.  Instead of printing out the actual gensym, Lisp will print
out a link back to the gensym declared in the objects declaration node.
The net effect is the same, however.

\section{Interpreting type expressions}

The currency of the type checker are type expressions.  Type expressions
are standard Lisp s-exprs with a special format.  Atomic Java types
are represented as atoms, with composite types being lists of the 
following forms:

\begin{verbatim}
(CLASS <class-name>)
(FUNCTION <return-type> <param-type-1> ...)
(ARRAY <element-type>)
\end{verbatim}
So, as an example, the declaration:
\begin{verbatim}
String greatest_string(String[] str_array, int array_size)
\end{verbatim}
would have the following type signature:
\begin{verbatim}
(FUNCTION (CLASS ``String'') (ARRAY (CLASS ``String'') INT)
\end{verbatim}



%%%%%%%%%%%%%%%%%%%%%%%%%%%%%%%%%%%%%%%%%%%%%%%%%%%%%%%%%%%%%%%%

\end{document}








